\documentclass[11pt,oneside,letterpaper]{article}
\usepackage{graphicx}
\usepackage{amssymb}
\usepackage{amsmath}
\usepackage{setspace}
\usepackage{layout}

\setlength{\evensidemargin}{0in}
\setlength{\topmargin}{0in}
\setlength{\oddsidemargin}{0in}
\setlength{\footskip}{1in}
\setlength{\textwidth}{6.5in}
\setlength{\textheight}{8in}
\title{Problem set 2}
\author{Lucas. S}
\date{January 2024}
\doublespacing
\newcommand{\ident}[0]{\\\indent}
\newcommand{\smatx}[4]{
  \Biggl[
  \begin{matrix}
  #1 & #2\\
  #3 & #4
  \end{matrix}
  \Biggr]
}
\newcommand{\modulo}[0]{\hspace{0.1em}\scalebox{0.5}{\textbf{\%}}\hspace{0.1em}}
\begin{document}


\maketitle
\section*{Problem 4}
\textbf{Question:}\\
Suppose $U$ is a subspace of $V$ and $V$ is a subspace of $W$. is $U$ a subspace of $W$?
\vspace*{8pt}\\
\textbf{Answer:}\\
$U$ is a subspace of $W$.
%
\vspace*{8pt}\\
\textbf{Proof:}\\
Due to the definition of subspaces, we can describe the relationship between the vector spaces' set of vectors like so:
\ident
$U\subseteq V\subseteq W$;\\
 therefore $U \subseteq W$. Furthermore, $U$ has the same standard operations as $V$, which has the same standard operations as $W$; so $U$ has the same standard operations as $W$.\\
With that established, the only remaining characteristic of a subspace is \textit{being} a vector space, which we know $U$ is, since it is a sub\textit{space} of $V$, and all subspaces must be vector spaces.
\vspace*{8pt}\\
\textbf{Conclusion:}\\
The relationship of a subspace can be carried through a chain. Firstly, we can show that $U$'s vectors are all elements of $W$. Next, we can show that there exists an \textit{equivalence} between $U$, $V$ and $W$; that is, they all share the same standard operations, and share the same defining properties of vector spaces.
\end{document}
